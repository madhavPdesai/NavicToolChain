\section{The tracking loop used in the IITB NAVIC SOC}

The process of tracking in the IITB NAVIC SOC consists of
the following flow of activity.

\begin{verbatim}
for each satellite S do {
  // F=carrier frequency
  // P=carrier phase
  // C=code delay
  // OK=1 if acquired, else 0
  F,P,C,OK := Acquire(S)
  CP := C           // initial prompt phase
  CE := C - CHIP/2  // initial early phase
  CL := C + CHIP/2  // initial late phase
  if(OK) {
    do {
       // in-phase and quadrature correlations
       // for prompt, early and late code delays,
       // using the carrier frequency F and phase P.
       Ip,Qp,Ie,Qe,Il,Ql := 
           ComputeCorrelations(S,F,P,CP,CE,CL);
       // Freq, phase, prompt, early, late code phases.
       F,P,CP,CE,CL :=
           RunTrackingLoop(Ip,Qp,Ie,Qe,Il,Ql,F,P,CP,CE,CL);
    } while S track not lost;
  }
} while TrackNotLost(S);
\end{verbatim}

The inner loop is executed every $1\ ms$. Let $T=0.001$ secods.
The tracking loop is responsible for computing
the updated values of 
\begin{itemize}
\item Carrier frequency F.
\item Carrier phase P.
\item Data code phase CP (hence CE, CL).
\end{itemize}
given the results of correlation computation based
on the previous values of F,P,CP.

We use $f$ to denote the carrier frequency,
$\phi$ to denote the carrier phase, and $\Delta_E, \Delta_P, \Delta_L$
to denote the data code phases CE, CP, CL shown above.  The tracking
loop actually consists of three loops, a frequency locked loop (FLL) to
update $f$, a phase locked loop (PLL) to update $\phi$, and a delay-locked
loop (DLL) to update $\Delta_P$.  Let the current index of the inner
loop execution be $k$ (we start with $k=1$)..

The FLL and PLL use an error value computed using the following
discriminator.
\begin{equation}
\theta(k) \ = \ \tan^{-1}(Ip(k)/Qp(k))
\end{equation}
The FLL is described by the following equation:
\begin{equation}
f(k+1)  =  ((0.055 \times 2 * \Pi * T) \times \theta(k))\ + \ f(k)
\end{equation}
The PLL is described by the following equations:
\begin{eqnarray*}
\phi_{R}(k+1) & = & \phi_R(k) \ + \ (f(k) \times 2 \times \Pi \times T) \\
\phi_{accum}(k+1)  & = & \phi_{accum}(k) \ + \ (0.75 \times \theta(k)) \\
\phi(k+1) & = & \phi_{accum}(k+1) + \phi_R(k+1)
\end{eqnarray*}
Together the FLL and DLL determine the values of $f$ and $\phi$ to
be used in the next correlation set.

The DLL uses the following discriminator:
\begin{eqnarray*}
E(k) & = & I_e(k)^2 + Q_e(k)^2\\
L(k) & = & I_l(k)^2 + Q_l(k)^2\\
\psi(k) & =  & (E(k) - L(k))/(E(k) + L(k)) 
\end{eqnarray*}
Based on the discriminator $\psi(k)$, the 
next value of the prompt code phase is calculated
using the following.
\begin{eqnarray*}
h(k+1) & = & (0.9 \times \psi(k)) + (0.1 \times h(k)) \\
P(k+1) & = & P(k) + ((h(k) > 0) ? -1 : 1)
\end{eqnarray*}

\vspace{1cm}
\begin{itemize}
\item It is assumed that $h(1) =\phi_R(1) = \phi_{accum}(1) = 0$.
\item Note that the tracking loop is entirely described by
software.
\end{itemize}

